\documentclass[a4paper, 11pt]{article}
\usepackage{polyglossia}
\setmainlanguage{english}
\setotherlanguage{swedish}
\usepackage{color}
\usepackage[parfill]{parskip}
\usepackage{siunitx}
\usepackage{graphicx}
\usepackage{float}
\usepackage{listings}
\usepackage{verbatim}
\usepackage{fancyvrb}
\usepackage{geometry}
\usepackage{tocloft}
\usepackage{fontspec}
\usepackage[numbers,square,super]{natbib}
\usepackage[numbib]{tocbibind}
%\usepackage{mathtools}
\usepackage[table]{xcolor}
\usepackage{soul}
\usepackage{nameref}
%\usepackage[addtotoc]{abstract}
%\usepackage[hyphens]{url}
\usepackage[bookmarks]{hyperref}
\usepackage[all]{hypcap}
\usepackage[hypcap]{caption}
%\usepackage[toc, page]{appendix}
\usepackage{pdfpages}

\cftsetindents{section}{0cm}{2cm}
\cftsetindents{subsection}{2cm}{1cm}
\cftsetindents{subsubsection}{3cm}{1cm}
%\cftsetindents{paragraph}{1.5cm}{0.5cm}

\makeatletter
\renewcommand\NAT@citesuper[3]{\ifNAT@swa
\if*#2*\else#2\NAT@spacechar\fi
\unskip\kern\p@\textsuperscript{\NAT@@open#1\if*#3*\else,\NAT@spacechar#3\fi\NAT@@close}%
   \else #1\fi\endgroup}
\makeatother


\let\strong\textbf 

\definecolor{ltblue}{rgb}{0.93,0.95,1.0}
\definecolor{dkblue}{rgb}{0,0,0.5}
\definecolor{dkyellow}{rgb}{0.5,0.5,0}
\definecolor{gray}{rgb}{0.5,0.5,0.5}
\definecolor{ltgray}{rgb}{0.9,0.9,0.9}
\definecolor{dkgreen}{rgb}{0,0.5,0}
\definecolor{dkmagenta}{rgb}{0.5,0,0.5}

\sethlcolor{ltgray}

\lstset{language=C}
\lstset{ %
    backgroundcolor=\color{white},   % choose the background color; you must add \usepackage{color} or \usepackage{xcolor}
    basicstyle=\scriptsize\ttfamily,        % the size of the fonts that are used for the code
    breakatwhitespace=false,         % sets if automatic breaks should only happen at whitespace
    breaklines=true,                 % sets automatic line breaking
    captionpos=b,                    % sets the caption-position to bottom
    commentstyle=\color{dkgreen},    % comment style
    deletekeywords={},               % if you want to delete keywords from the given language
    escapeinside={\%*}{*)},          % if you want to add LaTeX within your code
    extendedchars=true,              % lets you use non-ASCII characters; for 8-bits encodings only, does not work with UTF-8
    frame=none,                      % adds a frame around the code
    keepspaces=true,                 % keeps spaces in text, useful for keeping indentation of code (possibly needs columns=flexible)
    keywordstyle=\bfseries\color{dkblue},       % keyword style
    %identifierstyle=\color{dkyellow},
    directivestyle=\color{dkyellow},
    morekeywords={*,...},            % if you want to add more keywords to the set
    numbers=left,                    % where to put the line-numbers; possible values are (none, left, right)
    numbersep=5pt,                   % how far the line-numbers are from the code
    numberstyle=\tiny\color{gray},   % the style that is used for the line-numbers
    rulecolor=\color{black},         % if not set, the frame-color may be changed on line-breaks within not-black text (e.g. comments (green here))
    showspaces=false,                % show spaces everywhere adding particular underscores; it overrides 'showstringspaces'
    showstringspaces=false,          % underline spaces within strings only
    showtabs=false,                  % show tabs within strings adding particular underscores
    stepnumber=1,                    % the step between two line-numbers. If it's 1, each line will be numbered
    stringstyle=\color{dkmagenta},       % string literal style
    tabsize=4,                       % sets default tabsize to 2 spaces
    title=\lstname                   % show the filename of files included with \lstinputlisting; also try caption instead of title
}

\hypersetup{
    pdftitle={\doctitle},
    pdfauthor={\docauthor},
    pdfsubject={\docsubject},
    colorlinks=true,
    urlcolor=dkblue,
    linkcolor=black,
    citecolor=black,
    unicode=true,
}

\newfloat{source}{tbp}{ext}
\floatname{source}{Source code listing}

\patchcmd{\bibliography}{\section*}{\section}{}{}
\newcommand{\fig}[1]{\hyperref[fig:#1]{Figur \ref*{fig:#1}}}
\newcommand{\tab}[1]{\hyperref[tab:#1]{Tabell \ref*{tab:#1}}}
\newcommand{\src}[1]{\hyperref[src:#1]{Källkodsexempel \ref*{src:#1}}}
\newcommand{\term}[1]{\hyperref[term:#1]{#1}}
\newcommand{\defterm}[1]{\strong{#1}\label{term:#1}}

% New definition of square root:
% it renames \sqrt as \oldsqrt
\let\oldsqrt\sqrt
% it defines the new \sqrt in terms of the old one
\def\sqrt{\mathpalette\DHLhksqrt}
\def\DHLhksqrt#1#2{%
\setbox0=\hbox{$#1\oldsqrt{#2\,}$}\dimen0=\ht0
\advance\dimen0-0.2\ht0
\setbox2=\hbox{\vrule height\ht0 depth -\dimen0}%
{\box0\lower0.4pt\box2}}

%\renewcommand{\figurename}{Tabell }
\rowcolors{1}{white}{ltblue} % odd = white, even = lightblue

