\section{Design}

\subsection{Hardware}

\subsubsection{Microcontroller}
\subsubsection{Accelerometer}

\subsubsection{Antenna}
The antenna was designed as an LC tank circuit.
The inductor was created as a trace on the printed circuit board, the outer width and hight the same size as a credit card.

The inductor was first estimated [...]
After the printed circuit board had been created, the inductance of the antenna was measured with and LCR meter.
The measured inductance was 1.0 µH.

A tuning capacitor was calculated for the antenna resonant frequency of 13.56 MHz.
[...]

\subsubsection{Demodulation Circuitry}
For demodulating the incoming signal, a simple diode-capacitor-resistor based AM detector circuit was selected.
This circuit forms an envelope detector and low pass filter that will filter out the carrier wave and present the modulation wave on the input of the microcontroller.

The filter was initially set around 100 kHz to allow the 10 µs modulation pulses\cite{rfid-iso} through.
This however proved to be inadequate when analysing the signal using an oscilloscope.
The final filter was set around 450 kHz, with component values of 22 pF and 100 kΩ.

A 3.3V zener diode was added to the circuit to protect the microcontroller from high voltages.

\subsubsection{Modulation Circuitry}
For modulating the output signal, a diode in series with an N-channel mosfet, connected across the antenna, was used.

Initially, the diode was not present in the design.
This however caused serious problems, since the body diode of the transistor effectively short-circuited the antenna during half the duty cycle of the carrier wave.
With the diode added, the modulation is only performed during half of the duty cycle of the carrier wave, providing a weaker backscattered signal to the reader but was agreed upon as an acceptable compromise.

The modulation transistor is driven by an output pin of the microcontroller.

\subsubsection{Energy harvester}


\subsection{Software}

\subsubsection{Command decoding}
\subsubsection{CRC generation}
While CRC (a type of checksum) is designed to be reasonably quick to calculate in software,
a slow microcontroller will still struggle a bit to do it quickly. The microcontroller used in this
tag has a built-in accelerator for CRC computation. On paper, the CRC hardware inside the microcontroller
should be able to do any standard type of CRC up to CRC16, which is good since the ISO standard
implemented for RFID communication calls for a CRC16 checksum in every reply from the tag.

For the most part, configuring the CRC computation accelerator is mainly a case of loading the
CRC polynomial and a half-dozen parameters for things like bit shifting direction, whether or not to pad
out with zeroes when shifting, seed and starting values, which are all given in the ISO standard.
While the shifting direction can be set, the CRC
accelerator does not implement the lowest bit of the CRC polynomial, as any standard polynomial will always have
this bit set. As a result, the CRC accelerator inside the microcontroller cannot do reverse polynomials.
The ISO standard implemented in the tag for RFID communication requires a reverse polynomial CRC.

As a work-around, the polynomial provided by the ISO standard was reversed, and so was the bit shifting
direction as compared to what was called for in the ISO standard. The result is a correctly calculated
CRC, with exception for the bit order, which is completely reversed. A simple work-around for this is to
just shift out the bits in the opposite direction when modulating the reply, or to reverse the bits
in software before transmitting them. The latter is what ended up in the design. Additionally, the ISO standard
calls for the CRC to be transmitted as a one's complement of the calculted CRC, but this operation
can be trivially performed by the microcontroller.


\subsubsection{Response encoding}
