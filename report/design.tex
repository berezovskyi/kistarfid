\section{Design}

\subsection{Hardware}

\subsubsection{Microcontroller}
\subsubsection{Accelerometer}
\subsubsection{Energy harvester}
\subsubsection{Radio circuitry}

\subsection{Software}

\subsubsection{Command decoding}
\subsubsection{CRC generation}
While CRC (a type of checksum) is designed to be reasonably quick to calculate in software,
a slow microcontroller will still struggle a bit to do it quickly. The microcontroller used in this
tag has a built-in accelerator for CRC computation. On paper, the CRC hardware inside the microcontroller
should be able to do any standard type of CRC up to CRC16, which is good since the ISO standard
implemented for RFID communication calls for a CRC16 checksum in every reply from the tag.

For the most part, configuring the CRC computation accelerator is mainly a case of loading the
CRC polynomial and a half-dozen parameter for things like bit shifting direction, whether or not to pad
out with zeroes when shifting seed and starting values, which are all given in the ISO standard.
While the shifting direction can be set, the CRC
accelerator does not implement the lowest bit of the CRC polynomial, as any standard polynomial will always have
this bit set. As a result, the CRC accelerator inside the microcontroller cannot do reverse polynomials.
The ISO standard implemented in the tag for RFID communication requires a reverse polynomial CRC.

As a work-around, the polynomial provided by the ISO standard was reversed, and so was the bit shifting
direction as compared to what was called for in the ISO standard. The result is a correctly calculated
CRC, with exception for the bit order, which is completely reversed. A simple work-around for this is to
just shift out the bits in the opposite direction when modulating the reply, or to reverse the bits
in software before transmitting them. The latter is what ended up in the design. Additionally, the ISO standard
calls for the CRC to be transmitted as a one's complement of the calculted CRC, but this operation
can be trivially performed by the microcontroller.


\subsubsection{Response encoding}
