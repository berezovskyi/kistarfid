\section{Design}

\subsection{Hardware}

\subsubsection{Microcontroller}
\subsubsection{Accelerometer}

\subsubsection{Antenna}
The antenna was designed as an LC tank circuit.
The inductor was created as a trace on the printed circuit board, the outer width and hight the same size as a credit card.

The inductor was first estimated [...]
After the printed circuit board had been created, the inductance of the antenna was measured with and LCR meter.
The measured inductance was 1.0 µH.

A tuning capacitor was calculated for the antenna resonant frequency of 13.56 MHz.
[...]

\subsubsection{Demodulation Circuitry}
For demodulating the incoming signal, a simple diode-capacitor-resistor based AM detector circuit was selected.
This circuit forms an envelope detector and low pass filter that will filter out the carrier wave and present the modulation wave on the input of the microcontroller.

The filter was initially set around 100 kHz to allow the 10 µs modulation pulses\cite{rfid-iso} through.
This however proved to be inadequate when analysing the signal using an oscilloscope.
The final filter was set around 450 kHz, with component values of 22 pF and 100 kΩ.

A 3.3V zener diode was added to the circuit to protect the microcontroller from high voltages.

\subsubsection{Modulation Circuitry}
For modulating the output signal, a diode in series with an N-channel mosfet, connected across the antenna, was used.

Initially, the diode was not present in the design.
This however caused serious problems, since the body diode of the transistor effectively short-circuited the antenna during half the duty cycle of the carrier wave.
With the diode added, the modulation is only performed during half of the duty cycle of the carrier wave, providing a weaker backscattered signal to the reader but was agreed upon as an acceptable compromise.

The modulation transistor is driven by an output pin of the microcontroller.

\subsubsection{Energy harvester}


\subsection{Software}

\subsubsection{Command decoding}
\subsubsection{CRC generation}
\subsubsection{Response encoding}
