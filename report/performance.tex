\section{Performance}

As the developed tag is mostly in a stage of experimentation platform, a few parameters of its performance
was measured, mainly power consumption and reading distance. The reading distance was also compared to a
simple, commercial ISO 15693-2 (same standard as implemented in our experimental platform) tag. Specifically our
KTH library cards. Both the constructed tag and the experimental tag were tested using the same software with
the same 1/8 Watt reader using the same antenna.

\subsection{Reading Distance}
The reference point used, a commercial ISO 15693-2 tag, could reliably be read from 117 mm right above the
antenna, but any further than that would reliably fail. The tests where done holding the tag above the reader
antenna against a scale with only air between the antenna and the tag.

The constructed tag could with some difficulty be read at up to 7 mm, significantly shorter shorter than
the reference tag used. The data in table \ref{tab:readdistance} shows the time taken to read the tag at various distances.
The tag was read 5 times in rapid succession at each distance. For comparison, all successful reads against
the reference tag always took ≈ 330 ms.

\begin{table}
\centering
\begin{tabular}{| l | r | r | r | r | r |}
	\hline
	Distance & Read 1 & Read 2 & Read 3 & Read 4 & Read 5 \\ \hline
	0 mm & 365 ms & 665 ms & 336 ms & 351 ms & 351 ms \\ \hline
	1 mm & 685 ms & 671 ms & 669 ms & 669 ms & 339 ms \\ \hline
	2 mm & 1013 ms & 350 ms & 347 ms & 353 ms & 677 ms \\ \hline
	3 mm & 689 ms & 685 ms & 670 ms & 346 ms & 346 ms \\ \hline
	4 mm & 355 ms & 347 ms & 681 ms & 679 ms & 345 ms \\ \hline
	5 mm & 356 ms & 345 ms & 350 ms & 669 ms & 671 ms \\ \hline
	6 mm & 357 ms & 339 ms & 667 ms & 667 ms & 676 ms \\ \hline
	7 mm & 1352 ms & FAIL & 2680 ms & 2343 ms & 2669 ms \\ \hline
	8 mm & FAIL & FAIL & FAIL & FAIL & FAIL \\ \hline
\end{tabular}
\caption{Read times versus distance from antenna}
\label{tab:readdistance}
\end{table}

\subsection{Power Consumption}
Power consumption for the design was done by measuring the voltage over the energy storage capacitor
while reading the tag, and while measuring the current through the main rectifying diode.
At the furthest readable distance, 7 mm, the measured power consumption was 8.8 mA at 2.6 Volts (lowest peak while replying),
and about 10 mA at 3.1 Volts nominal. This places the power consumption somewhere between 22.8 mW and 31 mW. With the
accelerometer mounted to the tag specified at about 1.5 mW, the largest power consumers are likely the microcontroller and the
overvoltage protection zener diode.

When the tag is nudged out of range, the voltage does stay above the 2.2 volts required by the accelerometer
and the 1.8 volts required by the microcontroller, suggesting the design is back-link limited.
