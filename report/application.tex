\section{Project description and application}
The goal of the project is to use a general purpose microcontroller as an RFID soft-tag in the 13.56 MHz band.
The system will use inductive coupling to communicate with the reader and the tag will be completely passive, getting all its power from the reader's magnetic field.
Using a general purpose microcontroller enables more complex tasks to be performed by the RFID tag, such as reading a sensor or operating an actuator on the tag.

In the project, an accelerometer was attached to the tag, used to sense the orientation of the tag when held up against the reader.

\subsection{Application}
Soft-tags in general can potentially provide more flexibility than a system based on an off-the-shelf RFID tag, allowing processing to be done by the tag and interfacing with sensors and actuators, especially for smaller batches where cost probhibits the use of a custom ASIC design.

The particular system created for the project is more of a proof-of-concept, exploring the possiblities of a soft tag and interfacing with sensors from an RFID tag.
There are however some possible applications for attaching an accelerometer to an RFID tag.
Detecting the orientaion of a tag could be useful in shipment tracking, to verify that a container or box has been loaded with the correct side facing up, or for smart board games where the orientation of a game piece affects the gameplay.
